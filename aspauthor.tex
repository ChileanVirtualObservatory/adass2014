% This is the aspauthor.tex LaTeX file
% Copyright 2010, Astronomical Society of the Pacific Conference Series

\documentclass[11pt,twoside]{article}
\usepackage{asp2010}

\resetcounters

\bibliographystyle{asp2010}

\markboth{Jonathan Antognini, Mauricio Araya,  Mauricio Solar, Camilo Valenzuela, and Francisco Lira}{Jonathan Antognini, Mauricio Araya,  Mauricio Solar, Camilo Valenzuela, and Francisco Lira}

\begin{document}

\title{Evaluating a NoSQL alternative for Chilean Virtual Observatory Services}
\author{Jonathan Antognini,  Mauricio Araya, Mauricio Solar, Camilo Valenzuela, and  Francisco Lira
\affil{Universidad T\'ecnica Federico Santa Mar\'ia}
}

\begin{abstract}
Currently, the standards and protocols for data access in the Virtual Observatory architecture (DAL) are generally implemented with
relational databases based on SQL. In particular, the Astronomical Data Query Language (ADQL), language used by IVOA to represent 
queries to VO services, was created to satisfy the different data access protocols, such as Simple Cone Search. ADQL is based in 
SQL92, and has extra functionality implemented using PgSphere. An emergent alternative to SQL are the so called NoSQL databases, 
which can be classified in several categories such as Column, Document, Key-Value, Graph, Object, etc.; each one recommended for 
different scenarios. Within their notable characteristics we can find: schema-free, easy replication support, simple API, Big Data,
etc. The Chilean Virtual Observatory (ChiVO) is developing a functional prototype based on the IVOA architecture, with the following
relevant factors: Performance, Scalability, Flexibility, Complexity, and Functionality. Currently, it's very difficult to compare 
these factors, due to a lack of alternatives. The objective of this paper is to compare NoSQL alternatives with SQL through the 
implementation of a Web API REST that satisfies ChiVO's needs: a SESAME-style name resolver for the data from ALMA. Therefore, 
we propose a test scenario by configuring a NoSQL database with data from different sources and evaluating the feasibility of 
creating a Simple Cone Search service and its performance. This comparison will allow to pave the way for the application of 
Big Data databases in the Virtual Observatory.
\end{abstract}

\section{Introduction}
One of the requirements of the current ChiVO functional prototype
\cite{Antognini2013}, is the development of our own \emph{Name Resolver}
for astronomical objects that will be indexed by Alma. 
First we studied how other VO web services implement their name 
resolvers, and unsurprisingly, almost all of them use the CDS Sesame Name
Resolver \cite{}. 
Sesame uses SIMBAD data for resolving its queries, and send the results
back to the user in the form of a XML file.
SIMBAD itself offers a variety of search methods, one of them is
\emph{Coordinate Query}, similar to IVOA’s Simple Cone Search standard.

To have a similar database to SIMBAD, we have implemented two query methods that
SIMBAD offers: a simple criteria search and ConeSearch. For the second one, 
IVOA recommends using Postgresql along with Pgsphere module. 
However, we wanted to take a look in the new generation of databases engines, namely No-SQL databases, and compare the
performance of one of them against Postgresql using Pgsphere.


\section{Why No-SQL database}
    The huge amount of new data that ALMA observations will produce, obliges
    handling data more efficiently, so we could use the new technologies 
    in our advantage starting with how to data is stored for the variety
    of IVOA compliant services in ChiVO.
    
    To understand the differences between a SQL and No-SQL databases, we have
    to recall that the first one is based on ACID principle 
    (Atomicity, Consistency, Isolation and Durability) to ensure the
    transactions are made correctly, but those same properties
    makes SQL databases weak in horizontal scaling. On the other hand, No-SQL
    databases are based on the CAP principle (Consistency, Availability, Partition
    Tolerance), which makes them really useful for distributed environments, but 
    they doesn't support complex transactions or joins.

    To choose which No-SQL database could be useful for ChiVO, we base our
    decision on the fact that almost every
    IVOA Data Access Layer Protocol (DAL) need performing some type of spatial
    search. We have considered the Redis key-value database, due to its
    simplistic way to store data and ease to use, but it was very limited in
    their queries. Neo4j is a really complex graph database, which might help to
    understand the relationships between data, but it is hard to manipulate and slower 
    than other databases. Cassandra, a column database, its similar
    to SQL databases, but it has his own query language %[CV] no se como arreglar esto%
    . At last, Mongo document database, 
    has a built-in geospatial module, doesn't need a schema, and has reliable
    and thoughtful of documentation.

    For our tests we've chosen MongoDB, not only for the documentation, but also
    for the variety of query methods that it offers and the support of spatial indexes.


\section{MongoDB versus Postgresql performance}
    We prepared some tests for Mongo to compare it with Postgresql. From the user
    point of view, we tested \emph{read} capabilities of both databases, and
    for the developers points of view, we measured the \emph{write} operation
    performances.
    
    We started by testing both databases write capabilities, using a collection of 10.000 entries with data from SIMBAD database,
    and inserted them into the databases. We parsed the data into a JSON and a SQL file, 
    inserting data row by row into each the database, and measured the time until we 
    finished populating the database. The result was slightly superior
    performance of MongoDB with respect to PostgreSQL. (\emph{see fig \ref{fig:insert}})
    
 \begin{figure}
    \centering
    \begin{tabular}{lll}    
                        & MongoDB & Postgresql \\
    10.000 entries insertion & 3.12468409538 s  & 3.84547805786 sec
    \end{tabular}
    \caption{LIKE operand run-time comparison, using Regular expression in MongoDB; with Postgresql 8.4, and MongoDB 2.6.4}
    \label{fig:insert}
\end{figure}   
    
    With both databases populated, we were ready to test the read capabilities tests.
    
    The first read test was the SQL LIKE operator, because is particularly
    useful for the name resolver application. 
    MongoDB doesn't have LIKE operator, but we can use regular expressions to
    produce a similar query. Unfortunately, MongoDB was 7 times slower than
    PostgreSQL to make the query.(\emph{see fig\ref{fig:like}.})
\begin{figure}
    \centering
    \begin{tabular}{lll}    
                        & MongoDB & Postgresql \\
    LIKE operator query & 15.5ms  & 2.79225 ms
    \end{tabular}
    \caption{LIKE operand run-time comparison, using Regular expression in MongoDB; with Postgresql 8.4, and MongoDB 2.6.4}
    \label{fig:like}
\end{figure}

    The second read test was to measure how precise is MongoDB spatial queries
    compared to the same query in Postgresql.
    The result was the same amount of rows in all the 7776 queries made.
                
    At last, we tested the speed of spatial search, vital for conesearch and
    other DAL protocols, by making the same query with MongoDB 
    and Postgresql. We covered the northern celestial sky using a fixed radius,
    and compute the mean of the query's run-time. Then, we repeated the
    procedure with different radius values. MongoDB uses less than 0.1 ms in small queries, 
    but when we use a large radius, the time used
    in the query started to grow fast.(\emph{see fig\ref{fig:cone}.})
\begin{figure}
\centering

\begin{tabular}{lll}
 Radius &MongoDB &Postgresql\\
2 arcsec&0.0 ms &0.049387345679 ms\\
2 arcmin&0.0 ms &0.0496404320988 ms\\
5 degree&0.521604938272 ms&0.189644290123 ms\\ 
20 degree &8.76234567901 ms&2.05688194444 ms\\
35 degree &33.4328703704 ms& 7.83297376543 ms\\
50 degree & 55.912037037 ms&15.3577060185 ms
\end{tabular}


\caption{Conesearch query run-time comparison, using the same radius and changing the center of the cone}
\label{fig:cone}
\end{figure}    

\section{ChiVo's Name Resolver Implementation}
\begin{enumerate}
  
  % [CV] Aca no se como ligar que igual queriamos implementar la db en mongo 

\item  \textbf{Architecture.}  The Chivo architecture is composed of the Front-end web application, a Endpoint web application,
and a Data layer (see fig ). The Front-end is a Ruby on Rails application, that get the data from the Endpoint layer. 
The Data layer has is a GAVO Dachs data center, and has some of the DAL protocols implemented.
At last the Endpoint layer communicates the Front-end and the external applications with our Data layer and serve as a Load Balancer. (\emph{see fig \ref{fig3}})

\articlefigure{img/Endpoint_Diagram.eps}{fig3}{ChiVO Webservices Architecture}

\item  \textbf{The database.} Chivo's Name resolver uses a Mongo database, with the built-in 2sphere indexes 
to make the Conesearch queries. 

\item  \textbf{The Endpoint’s web interface.} ChiVO Endpoint uses Flask, to have a lightweight application; 
Flask is a Microframework written in python, we also can handle the data with astropy
libriaries, to make the mongo query and parse the results to VoTable so we can fulfill IVOA Conesearch standard.

\item \textbf{Name Resolver Interface.} We handle the request in the Flask Endpoint,
then make a query to the MongoDB database containing the object name, at the end we return the user the coordinates of the object 
in a JSON format. The implemented prototype is available in \url{http://dachs.lirae.cl/name_resolver/(name)} for the name resolver
and in \url{http://dachs.lirae.cl/bib/conesearch?RA=(right_ascension)&DEC=(declination)&SR=(radius)}.

\end{enumerate}

\section{What's next}
    Implementing a Conesearch using new technologies has proved to be really
simple with MongoDB, because it has proper documentation, 
and a growing community, in contrast of implementing it with PostgreSQL and
PgSphere module. Moreover,  PgSphere is getting deprecated, has nearly no active
community, and doesn't work with the newer versions of PostgreSQL. 
However,  MongoDB lacks performance in a single server, which makes it not yet
wise to change from PostgreSQL to MongoDB.

The next step is to update PgSphere to use the new features of PostgreSQL 9.3+, like JSON datatype and replication improvements.
Test MongoDB clusters read capabilities, develop indexation for column based databases, that offer better read capabilities than document based databases.


% [CV] No se como poner las referencias %
\bibliography{aspauthor}

\end{document}
